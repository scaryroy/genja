\documentclass[journal,a4paper]{IEEEtran}

\usepackage{texments}
\usepackage{algpseudocode}
\usepackage[english]{babel}
\usepackage{hyperref}

\usestyle{default}

\hyphenation{li-ne-ar-i-za-tion}

\begin{document}

\title{Java with Generators}
\author{Tony~Young, Student ID: 5383914, University of Auckland}%

\maketitle

\begin{abstract}

This report details the implementation of \emph{generators} in Java -- a specialized case of
continuations that allow the execution of code to be paused within a method body at specific
\emph{yield} points. It introduces \emph{linearization} as the primary technique for achieving
this, with additional supplementary algorithms to transform code to adhere to the semantics of
Java.

\end{abstract}

\section{Introduction}

Generators are a language feature found in various modern languages. They enable sequential code to
be paused at specific \emph{yield} points, making it useful for tasks such as iterating through a
binary tree or creating lazy, infinite lists (\`a la Haskell) without having to explicitly keep
track of where execution is.

This implementation of generators extends Java syntax in two ways:

\begin{itemize}
\item An extension to the Java method syntax. Before the method name, we allow an optional
      \texttt{*} token that signifies the method is a generator that will yield values of the type
      specified for the method.

\item The introduction of a \texttt{yield} statement. It is allowed only in starred generator
      methods and takes an expression that will be returned a subsequent execution of the
      generator.
\end{itemize}

For simplicity's sake, generators in this implementation are unidirectional and only yield values
out (such as in C\#), unlike the generators in Python or ECMAScript Harmony can have values ``sent"
to them.

\section{Transformation}

The code is transformed in a multi-pass fashion -- the bulk of transformation is done during the
linearization process (section~\ref{section:linearization}), but some pre- and post-processing
passes are employed to make linearization simpler.

\subsection{Loop Desugaring}

The first phase of generator transformation is loop desugaring. In this phase, we transform all
loops into a \emph{canonical form}, which is proposed to be the following:

\begin{pygmented}{java}
for (;;) { /* loop body */ }
\end{pygmented}

The canonical loop form represents an infinite loop -- however, the loop body may contain
\texttt{break} and \texttt{continue} statements to mimic sugared loop forms.

As such, automated conversion can be employed from other forms of loops (such as \texttt{while},
\texttt{for} and \texttt{do}). For a full list of conversions from canonical loop forms, refer to
appendix~\ref{appendix:canonical-loop-forms}.

In the special case of a non-canonical \texttt{for} loop, we may need to move an outer label into
the inner canonical \texttt{for} loop. This is facilitated by remembering the position of labels
before entering their statement nodes as well as which statement node was entered then, during
processing of the statement node, we can indicate that we have moved the label into a loop and that
the label needs to be removed.

\subsection{Scope Mangling}

Java employs block-level scoping for variables -- for example, variables declared inside an
\texttt{if} block will not be visible upon exit of the \texttt{if} block. Due to linearization
(section~\ref{section:linearization}), we cannot preserve blocks and, consequently, we cannot use
variable names in the form they are suggested by the code, for example:

\begin{pygmented}{java}
{
    int i = 0;
}
{
    float i = 0.0f;
}
\end{pygmented}

\texttt{i} is typed as an \texttt{int} in the first scope, but \texttt{float} in the second. If
the statements are placed into a single block from both the blocks, we will find that we have a
variable name clash, as well as the variable \texttt{i} having its type changed.

As such, we mangle variable names in deeper scopes to ensure they don't conflict with variable
names in higher scopes. Figure~\ref{algorithm:scope-mangling} details a basic algorithm for name
mangling. We also perform partial semantic analysis of variable declaration at this point, ensuring
that local variables are not redeclared. We maintain a stack of block scopes that represents the
current block of code we're in, as well as all the ancestral blocks. The block scopes contain
symbol tables that keep track of variable declarations in this scope -- if we find that a variable
has been redeclared either in this scope or any ancestral scope, this is a semantic error.

\begin{figure}
\begin{algorithmic}
\Function{ScopeMangle}{node}
    \State id $\gets$ 0
    \ForAll{b $\in$ node's blocks}
        \ForAll{d $\in$ b's declarations}
            \State \Call{Prepend}{id, d's variable name}
        \EndFor

        \ForAll{e $\in$ b's expressions}
            \ForAll{n $\in$ e's referenced names}
                \State \Call{Prepend}{n's block's id, n}
            \EndFor
        \EndFor
        \State id $\gets$ id + 1
    \EndFor
\EndFunction
\end{algorithmic}
\caption{Scope mangling algorithm.}
\label{algorithm:scope-mangling}
\end{figure}

In practice, the block ID is not a simple monotonically increasing number. For the implementation
provided, the blocks are first numbered off by their offset to the start of the parent. The
variables are then mangled using successive prepending of block numbers from the block the variable
is declared in to the root of the method -- for example, in the example given earlier, the three
variables will be mangled to \texttt{s0\$i}, \texttt{s1\$i}. If a variable \texttt{j} was declared
inside the second block inside a block, then it will be mangled to \texttt{s1\$s0\$j}.

\subsection{Linearization} \label{section:linearization}

The linearization (pertaining to loop linearization) pass processes the control flow graph and
transforms it into a state machine. This is required because we can then create states that
represents the pausing of the state machine, at given yield points. Additionally, linearized
control structures enable us to perform arbitrary jumps into the code -- such as jumps into code
after the yield point to simulate resumes.

Before linearization, we employ a node marking scheme to avoid excessive linearization. The
algorithm is detailed in figure~\ref{algorithm:mark-for-linearizing} details which nodes are
selected to be linearized. In particular, any node containing a \texttt{yield}, \texttt{break} or
\texttt{continue} must be linearized, as well as all of their ancestors.

\begin{figure}
\begin{algorithmic}
\Function{MarkLinearization}{node}
    \If{node $\in$ yields $\cup$ breaks $\cup$ continues $\cup$ returns}
        \State \Call{Mark}{node}
        \State \Return True
    \EndIf
    \ForAll{n $\in$ node's children}
        \If{\Call{MarkLinearization}{n}}
            \State \Call{Mark}{n}
        \EndIf
    \EndFor
    \If{nodes were marked}
        \State \Return True
    \Else
        \State \Return False
    \EndIf
\EndFunction\end{algorithmic}
\caption{Linearization marking algorithm.}
\label{algorithm:mark-for-linearizing}
\end{figure}

Once we have the nodes for linearization, we apply the straightforward linearization transforms to
control flow as described in appendix~\ref{appendix:linearizations}. States are in the form of
\texttt{case} statements, which denote a state number that can be jumped to by setting the
appropriate state number and \texttt{break}ing from the case.

For yield statements, we create a state for resuming execution when we resume the state machine. In
the previous state, we create a deferred jump to the resume state, set the yielded value as the
current value emitted by the state machine and then return \texttt{true} from the state machine to
signify that the current value has changed.

We keep track of two things during linearization: labels and loops. A loop can be considered an
implicit label that is given a name using the enumeration order. A label refers to a statement that
can be jumped to, but only from a child statement of the labeled statement. With loop and label
state, the following conditions apply to the linearization transformer:

\begin{itemize}
\item When the transformer encounters a label statement, we push the label onto the stack of labels
with its name, as well as onto the map of labels indexed by string. The label will also have the
start state associated with it, i.e. the point we will jump to if we \texttt{continue} to it. After
we process the label's child statement, we know the state at which the statement ends so we set
that as the point that we jump to if we attempt to \texttt{break} to it. During transformation of
child \texttt{break} and \texttt{continue} statements, we verify that we only ever attempt to
\texttt{break} or \texttt{continue} to labels found in parent node. If a label is directly adjacent
to a loop, the label is marked as being a loop label (and consequently allows \texttt{continue}s).

\item When the transformer encounters a canonical loop, we repeat the same procedure we did with
the labels with the exception that we do not add them to the map of labels, nor do we name the loop
labels. The transformer will then transform any label-less \texttt{break} and \texttt{continue}
statements it finds into labeled ones, using special labels prefixed with \texttt{.loop}, followed
by the loop number as dictated by enumeration order. We only allow \texttt{continue} statements
to jump to an ancestral loop block.
\end{itemize}

After all the code has been linearized, we create a trap state that always deferred-jumps to itself
and returns \texttt{false}. This ensures we don't execute random code in the state machine.

\subsubsection{Labeled Jump Resolution}

The initial linearization pass should have already verified that we aren't performing jumps to
non-ancestral blocks. As such, we just look up labels in the linearization context and generate
jumps to them for \texttt{break} statements. We do the same for \texttt{continue} statements,
except we use the continue points instead of the break points.

At this point, we expect no more \texttt{break}s or \texttt{continue}s that haven't been
dereferenced.

\subsection{Wrapping}

Linearization will have created a full state machine, which must then be wrapped in a
\texttt{switch} statement wrapped in an infinite loop to facilitate jumping between states. This
implementation supplies a runtime class, \texttt{genja.rt.Generator}, that wraps the generator to
conform to Java's iterator interface. This class adopts slightly different semantics from standard
iterators, due to the fact that generators are slightly different to Java iterators:

\begin{itemize}
\item A \texttt{hasNext()} call on a \texttt{genja.rt.Generator} will resume the underlying state
machine and set the last state machine success result if and only if the generator's stale flag is
set to true. If the stale flag is not set, we simply return the last state machine success result.

\item A \texttt{next()} call on a \texttt{genja.rt.Generator} will make a call to
\texttt{hasNext()} if and only if the generator's stale flag is set to true. Otherwise, we return
the current value yielded from the state machine and set the stale flag to true.
\end{itemize}

\section{Known Issues}

Due to the time constraints of this assignment, the following features have not been implemented.

\begin{itemize}
\item try-catch-finally. This involves catching exceptions at every state that is surrounded by
      a try block -- possibly multiple try blocks. Additional machine states are generated for
      catches, and an unconditional jump needs to be generated to finally states.

\item Duplicate state elimination. The linearization pass generates states at each point where
      execution could jump to or be resumed at. These are often duplicated and can be eliminated
      after linearization.

\item Unreachable jump elimination. Occasionally, the linearization pass generates unconditional
      jumps directly after unconditional jumps. These, of course, have no effect and can be
      removed.

\item Selective loop desugaring. Currently, all loops are desugared. This could be eliminated with
      an additional annotation pass before desugaring, and making the desugaring pass aware of
      annotations.

\item Selective loop linearization. Currently, all loops (except for infinite ones) are linearized
      completely. This is because the node annotator chooses to mark \texttt{break} and
      \texttt{continue} as statements that require a linearized block -- as these statements can
      jump into linearized code from a non-linearized block.
\end{itemize}

\section{Conclusion}

Through a multi-pass compilation process, we can effectively generate state machines in Java via
linearization of control flow. These can then be in turn transformed into generators and made
compatible with Java's iterators to facilitate various interesting constructs, such as infinite
lazy lists and abandonable loop processing from yield points.

\appendix
\subsection{Canonical Loop Forms} \label{appendix:canonical-loop-forms}

Here are the canonical loop forms for various types of loops.

\subsubsection{\texttt{while}}

\begin{pygmented}{java}
label: while (pred) {
    body;
}
\end{pygmented}

becomes:

\begin{pygmented}{java}
label: for (;;) {
    if (!pred) break; body;
}
\end{pygmented}

\subsubsection{\texttt{do}}

\begin{pygmented}{java}
label: do {
    body;
} while (pred)
\end{pygmented}

becomes:

\begin{pygmented}{java}
label: for (;;) {
    body; if (!pred) break;
}
\end{pygmented}

\subsubsection{\texttt{for} (non-canonical loop)}

\begin{pygmented}{java}
label: for (init; pred; update) {
    body;
}
\end{pygmented}

becomes:

\begin{pygmented}{java}
{
    init;
    label: for (;;) {
        if (!pred) break; body;
        update;
    }
}
\end{pygmented}

\subsection{Linearizations} \label{appendix:linearizations}

Here are linearized forms for various control structures, as well as \texttt{yield}. The case
numbers only intend to denote flow.

\subsubsection{\texttt{if}}

\begin{pygmented}{java}
preamble;
if (pred) {
    consequent;
} else {
    alternate;
}
postamble;
\end{pygmented}

becomes:

\begin{pygmented}{java}
case 0:
    preamble;
    if (pred) {
        state = 1;
        break;
    } else {
        state = 2;
        break;
    }
case 1:
    consequent;
    state = 3;
    break;
case 2:
    alternate;
case 3:
    postamble;
\end{pygmented}

\subsubsection{\texttt{for} (canonical loop)}

\begin{pygmented}{java}
preamble;
for (;;) {
    body;
}
postamble;
\end{pygmented}

becomes:

\begin{pygmented}{java}
case 0:
    preamble;
case 1:
    body;
case 2:
    state = 1;
    break;
case 3:
    postamble;
\end{pygmented}

\subsubsection{\texttt{switch} (with \texttt{default})}

\begin{pygmented}{java}
preamble;
switch (x) {
    case A:
        bodyA;
        break;

    case B:
        bodyB;

    case C:
    case D:
        bodyD;
        break;

    default:
        bodyDefault;
}
postamble;
\end{pygmented}

becomes:

\begin{pygmented}{java}
case 0:
    preamble;
    switchJump = false;
    switch (x) {
        case A:
            switchJump = true;
            state = 1;
            break;
        case B:
            switchJump = true;
            state = 2;
            break;
        case C:
        case D:
            switchJump = true;
            state = 3;
            break;
        default:
            switchJump = true;
            state = 4;
            break;
    }
    if (!switchJump) {
        state = 5;
        break;
    }
case 1:
    bodyA;
    state = 6;
    break;
case 2:
    bodyB;
case 3:
case 4:
    bodyD;
    state = 6;
    break;
case 5:
    bodyDefault;
case 6:
    postamble;
\end{pygmented}


\subsubsection{\texttt{switch} (without \texttt{default})}

\begin{pygmented}{java}
preamble;
switch (x) {
    case A:
        bodyA;
        break;

    case B:
        bodyB;

    case C:
    case D:
        bodyD;
        break;
}
postamble;
\end{pygmented}

becomes:

\begin{pygmented}{java}
case 0:
    preamble;
    switchJump = false;
    switch (x) {
        case A:
            switchJump = true;
            state = 1;
            break;
        case B:
            switchJump = true;
            state = 2;
            break;
        case C:
            switchJump = true;
            state = 3;
            break;
        case D:
            switchJump = true;
            state = 4;
            break;
    }
    if (!switchJump) {
        state = 4;
        break;
    }
case 1:
    bodyA;
    state = 6;
    break;
case 2:
    bodyB;
case 3:
case 4:
    bodyD;
    state = 5;
    break;
case 5:
    postamble;
\end{pygmented}

\subsubsection{\texttt{yield}}

\begin{pygmented}{java}
preamble;
yield x;
postamble;
\end{pygmented}

becomes:

\begin{pygmented}{java}
case 0:
    preamble;
case 1:
    state = 2;
    current = x;
    return true;
case 2:
    postamble;
\end{pygmented}

\end{document}
